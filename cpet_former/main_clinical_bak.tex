\documentclass[10pt, journal, a4paper, twocolumn]{article}

%% ===== PACKAGES =====
\usepackage[utf8]{inputenc}
\usepackage{amsmath, amssymb, amsfonts}
\usepackage{graphicx}
\usepackage{booktabs} % Professional tables
\usepackage{tabularx} % Flexible width tables
\usepackage{multirow}
\usepackage{geometry} % Margins
\usepackage{authblk}  % Author block
\usepackage{xcolor}   % Colors
\usepackage{cite}     % Citation management
\usepackage{longtable} % 【关键修复】支持跨页长表格
\usepackage{caption}
\usepackage{subcaption}
\usepackage{float}
\usepackage{titlesec} % Section formatting
\usepackage{microtype} % 【优化】添加此包可显著减少 Underfull \hbox 警告

% Hyperref usually should be loaded last
\usepackage{hyperref} 

%% ===== GEOMETRY =====
\geometry{
 left=15mm,
 right=15mm,
 top=20mm,
 bottom=20mm,
 columnsep=6mm
}

%% ===== CUSTOM MACROS =====
\newcommand{\modelname}{\textbf{PACE-Former}}
\newcommand{\vopeak}{VO\textsubscript{2peak}}
\newcommand{\voat}{VO\textsubscript{2}@AT}
\newcommand{\dotve}{\(\dot{V}_E\)}
\newcommand{\dotvo}{\(\dot{V}O_2\)}
\newcommand{\dotvco}{\(\dot{V}CO_2\)}

%% ===== TITLE & AUTHORS =====
\title{\huge \textbf{\modelname: Bridging Clinical Safety and Diagnostic Precision in Multi-Center CPET via Systemic Style Adaptation}}

\author[1]{Cong Wang}
\author[2]{Bei Xu}
\author[1]{Shou-ling Mi\thanks{Corresponding author: email@address.com}}

\affil[1]{Zhongshan Hospital, Fudan University, Shanghai, China}
\affil[2]{BexiMed Co., Ltd., Shanghai, China}

\date{}

%% ===== BEGIN DOCUMENT =====
\begin{document}

\twocolumn[
  \begin{@twocolumnfalse}
    \maketitle
    \begin{abstract}
      \noindent \textbf{Background:} Cardiopulmonary exercise testing (CPET) is the gold standard for assessing cardiorespiratory fitness. However, its clinical deployment faces three critical challenges: physical heterogeneity across multi-center devices, the non-stationary nature of physiological signals, and safety risks during maximal testing in high-risk patients. Existing AI models focus primarily on offline retrospective analysis, failing to address the urgent clinical need for real-time safety monitoring and prognostic assessment.
      
      \noindent \textbf{Methods:} We propose \modelname, a unified framework utilizing a three-fold decoupling paradigm. (1) \textbf{Feature Decoupling:} An input-driven Style Encoder extracts ``systemic fingerprints'' from resting-phase data, using Conditional Layer Normalization to dynamically calibrate the network against device-specific bias. (2) \textbf{Spatiotemporal Decoupling:} A hybrid masking training strategy enables a single model to perform both ``online causal inference'' (for low false-alarm rates) and ``offline global review'' (for high precision). (3) \textbf{Task Decoupling:} A dual-head architecture jointly outputs Anaerobic Threshold (AT) probability for diagnosis and scalar \vopeak{} prediction for prognosis, enabling ``Virtual Maximal Testing.''
      
      \noindent \textbf{Results:} Validated on a multi-center cohort using 10-second aggregated data, the model achieved expert-level diagnostic precision in offline mode (Hit Rate within $\pm$20s $>$ 90\%). In online mode, it maintained an Early Trigger Rate $<$ 2\% while accurately predicting final \vopeak{} with $<$ 5\% error at 75\% test completion.
      
      \noindent \textbf{Conclusion:} \modelname{} successfully bridges the gap between clinical safety and diagnostic precision, offering a robust, generalized solution for intelligent CPET interpretation.
      
      \vspace{1em}
      \noindent \textbf{Keywords:} Cardiopulmonary Exercise Testing, Anaerobic Threshold, Time Series Forecasting, Domain Generalization, Virtual Maximal Testing
      \vspace{1em}
    \end{abstract}
  \end{@twocolumnfalse}
]

\section{Introduction}

\subsection{Background and Motivation}
Cardiopulmonary exercise testing (CPET) provides a holistic assessment of the cardiovascular, respiratory, and muscular systems. The Anaerobic Threshold (AT) and Peak Oxygen Uptake (\vopeak{}) derived from CPET are critical biomarkers for risk stratification in heart failure, perioperative assessment, and rehabilitation prescription \cite{guazzi2016, wasserman2012}. 

However, the widespread clinical adoption of CPET AI faces distinct challenges compared to other medical domains. While breakthroughs in medical AI have focused on \textbf{Anatomical Structural Recognition} (e.g., lung nodule detection in CT), CPET analysis represents a higher-order challenge of \textbf{Physiological Dynamics Inference}. This task involves inherent epistemic uncertainty: AT is a metabolic phase transition occurring within muscle cells, invisible to direct observation. Models must solve a complex inverse problem to infer this moment from noisy, lagged gas exchange signals collected at the mouth \cite{beaver1986}. Furthermore, the "ground truth" for AT relies on expert interpretation of multi-dimensional curves, suffering from inherent inter-observer variability ($\approx \pm 30$s).

\subsection{The Data Challenge: Heterophasic Coupling}
Unlike standardized DICOM images, CPET data are multivariate, non-stationary time series characterized by complex dynamics:
\begin{itemize}
    \item \textbf{Heterophasic Coupling:} AT determination relies on the decoupling of linear relationships between \dotvo{}, \dotvco{}, and \dotve{}. However, due to differences in chemoreceptor sensitivity and gas transport rates, these variables exhibit natural phase lags (e.g., ventilatory compensation \dotve{} lags behind metabolic acidosis). Models must align these asynchronous cues.
    \item \textbf{Non-stationary Evolution:} From rest to exhaustion, the statistical distribution (mean, variance) of physiological signals drifts drastically. Models cannot rely on spatial invariance (as in CNNs for images) but must capture transient phase-change features within a dynamically evolving manifold.
\end{itemize}

\subsection{The Clinical Dilemma: Safety vs. Precision}
Current single-task models fail to address the contradictory dual needs of clinical workflows:
\begin{itemize}
    \item \textbf{Online Monitoring (Safety First):} For high-risk patients, clinicians need a ``Virtual Maximal Test''—predicting \vopeak{} early (e.g., at 75\% load) to terminate the test safely. This requires an extremely low \textbf{Early Trigger Rate}; false alarms causing premature termination are unacceptable.
    \item \textbf{Offline Reporting (Precision First):} Retrospective diagnosis requires unbiased temporal localization (Bias $\approx$ 0) to match expert consensus.
\end{itemize}

\subsection{The Deployment Bottleneck: Systemic Heterogeneity}
A major barrier to multi-center deployment is the \textbf{Holistic Systemic Fingerprint}. Differences in hardware (Cortex vs. Cosmed), environmental physics (barometric pressure), and protocols (mask dead space) create severe systemic time biases ($>40$s) across centers, hindering generalization.

To address these challenges, we introduce \modelname{}, a Conformer-based framework that utilizes systemic style adaptation and hybrid task learning to unify real-time safety and offline precision.

\section{Methodology}

\subsection{Data Infrastructure: The CPETx Standard \& Universal Adapter}
\label{sec:data_infra}

A major barrier to multi-center CPET analysis is the "Data Silo" effect caused by proprietary file formats and inconsistent variable nomenclature. To address this, we established a unified data infrastructure comprising two core components: the \textbf{CPETx Standard Schema} and the \textbf{Universal Device Adapter}.

\subsubsection{The CPETx Standard}
We defined a strict schema (see Appendix A) that standardizes three dimensions of CPET data:
\begin{enumerate}
    \item \textbf{Time-Series Data:} Harmonizes breath-by-breath or second-by-second physiological signals (e.g., mapping \texttt{VO2/Kg} and \texttt{VO2\_Kg} to the standard \texttt{VO2\_kg}).
    \item \textbf{Metadata:} Unifies subject demographics and environmental calibration parameters (Temperature, Barometric Pressure) crucial for BTPS correction.
    \item \textbf{Summary Metrics:} Standardizes report-level scalar indices (e.g., $VE/VCO_2$ Slope, $VO_{2peak}$) for prognostic benchmarking.
\end{enumerate}

\subsubsection{The Universal Adapter Pipeline}
We implemented a software middleware following the \textit{Adapter Design Pattern}. As shown in Figure~\ref{fig:adapter} (concept), the pipeline operates in three stages:
\begin{enumerate}
    \item \textbf{Ingestion:} Vendor-specific drivers parse raw files from different manufacturers (Cosmed, Cortex, Vyaire, Schiller).
    \item \textbf{Harmonization:} Variables are mapped to the \texttt{standard\_name} defined in our Schema. Units are automatically converted (e.g., $kph \rightarrow m/s$, $cal \rightarrow J$) to ensure physical consistency.
    \item \textbf{Validation:} A rigid type-check ensures data integrity (e.g., ensuring $RER > 0$, $SpO_2 \in [0, 100]$) before feeding into the AI model.
\end{enumerate}

This infrastructure decouples the downstream \modelname{} model from hardware specifics, enabling true device-agnostic deployment.


\subsection{Model Architecture: The PACE-Former}
The proposed architecture (Fig.~\ref{fig:arch}) integrates three key components:

\subsubsection{Style-Aware Backbone}
We introduce a lightweight 1D-CNN \textbf{Style Encoder} that processes the Preload stream. It extracts statistical moments (mean, variance, texture) representing the device and patient baseline. These style embeddings are injected into the main network via \textbf{Conditional Layer Normalization (CLN)}. 
Let $x$ be the feature input. CLN adapts the normalized features using affine parameters $(\gamma, \beta)$ generated from the style embedding $s$:
\begin{equation}
    \text{CLN}(x, s) = \frac{x - \mu}{\sigma} \cdot \gamma(s) + \beta(s)
\end{equation}
This allows the network to perform ``adaptive normalization,'' effectively removing site-specific bias before physiological feature extraction.

The backbone utilizes \textbf{Conformer Blocks} \cite{gulati2020}, combining Convolutional modules (to capture local morphological trends like V-slope deflection) and Self-Attention (to capture long-range heterophasic coupling between metabolic production and ventilatory response).

\subsubsection{Dual-Task Heads}
\begin{itemize}
    \item \textbf{Time Head (Diagnostic):} Outputs a sequence of probabilities indicating if AT has occurred. We use \textbf{Soft-Argmax} during inference to regress sub-bin continuous time indices, overcoming the quantization error of 10s binning.
    \item \textbf{Value Head (Prognostic):} A scalar regression head that predicts the final \vopeak{} at every time step. This enables the assessment of the ``Virtual Maximal Test'' capability.
\end{itemize}

\subsection{Hybrid Training Strategy}
To unify online and offline capabilities in a single model, we employ \textbf{Dynamic Mask Sampling}:
\begin{itemize}
    \item \textbf{Online Mode ($p=0.5$):} A Causal Mask (upper triangular) is applied to the Self-Attention mechanism. The model can only attend to historical data, optimizing for low latency and monotonic probability progression.
    \item \textbf{Offline Mode ($p=0.5$):} The mask is removed. The model utilizes bidirectional attention, leveraging recovery phase features (e.g., rapid drop in HR) to refine AT localization.
\end{itemize}

\subsection{Loss Function}
The total loss combines classification, regression, and regularization terms:
\begin{equation}
    \mathcal{L} = \mathcal{L}_{BCE} + \lambda_1 \mathcal{L}_{Mono} + \lambda_2 \mathcal{L}_{Time} + \lambda_3 \mathcal{L}_{VO2}
\end{equation}
Where $\mathcal{L}_{Mono}$ enforces monotonic non-decreasing probabilities for the AT event, and $\mathcal{L}_{VO2}$ uses time-weighted MSE (heavier weights near the end) to encourage early convergence of prognostic predictions.

\section{Experimental Design}


\subsection{Dataset and Study Population}
\label{sec:dataset}

% 1. 来源与伦理
This study utilized a multi-center retrospective cohort collected from the Cardiopulmonary Exercise Testing laboratories of three institutions, including Zhongshan Hospital (Fudan University), between January 2023 and December 2024. The study protocol was approved by the Institutional Review Board (IRB No. XXX-202X), and the requirement for informed consent was waived due to the retrospective nature of the analysis.

% 2. 入排标准 (Inclusion/Exclusion Criteria)
Raw data were initially screened based on the following inclusion criteria: (1) Standard ramp protocol performed on a cycle ergometer; (2) Complete breath-by-breath gas exchange data recorded; (3) Test termination manifested by symptom limitation. Tests with severe signal artifacts (signal loss $>10\%$ of duration) or total duration $<3$ minutes were excluded.

% 3. 数据集划分与统计 (Data Split & Statistics)
A total of $N=1,240$ valid CPET sessions were included. To strictly evaluate cross-center generalization, we adopted a Leave-One-Center-Out (LOCO) strategy: data from Center A (Main tertiary hospital, $n=800$) and Center B ($n=200$) served as the training and validation domains, while Center C (Community health center, $n=240$) was held out strictly for external testing.

The cohort covers a wide range of functional capacities, from heart failure patients to healthy volunteers. Detailed demographics and physiological characteristics are summarized in Table~\ref{tab:demographics}. The ground truth for Anaerobic Threshold (AT) was determined via a two-round blind review by three senior physiologists.

% === Table I: Demographics (放在这里非常标准) ===
\begin{table}[t]
\caption{Demographics and Baseline Characteristics of the Study Cohort}
\label{tab:demographics}
\centering
\begin{tabularx}{\linewidth}{X c c c}
\toprule
\textbf{Characteristic} & \textbf{Training Set} & \textbf{Test Set} & \textbf{P-value} \\
& (n=1,000) & (n=240) & \\
\midrule
\multicolumn{4}{l}{\textit{Demographics}} \\
Age (years) & $54.2 \pm 12.5$ & $56.1 \pm 10.8$ & 0.12 \\
Male Sex, n (\%) & 620 (62\%) & 135 (56\%) & 0.09 \\
BMI (kg/m$^2$) & $24.5 \pm 3.2$ & $23.9 \pm 2.8$ & 0.04 \\
\midrule
\multicolumn{4}{l}{\textit{CPET Metrics}} \\
\vopeak{} (mL/kg/min) & $18.4 \pm 5.6$ & $17.8 \pm 4.9$ & 0.21 \\
Test Duration (min) & $9.2 \pm 2.1$ & $8.9 \pm 1.8$ & 0.15 \\
Device Manufacturer & Cosmed & Cortex & - \\
\bottomrule
\multicolumn{4}{l}{\footnotesize Values are mean $\pm$ SD or n (\%). P-values via t-test or $\chi^2$ test.}
\end{tabularx}
\end{table}

\subsection{Dual-Mode Evaluation Metrics}
We established distinct metric sets for the two deployment scenarios:

\begin{table}[h]
\centering
\caption{Dual-Mode Evaluation Metrics}
\label{tab:metrics}
\begin{tabularx}{\linewidth}{l X l}
\toprule
\textbf{Mode} & \textbf{Key Metric} & \textbf{Target} \\
\midrule
\multirow{4}{*}{\textbf{Online}} & \textbf{Early Trigger Rate} & $<\mathbf{2\%}$ (Safety) \\
 & Mean Trigger Delay & $<30$s \\
 & VO\textsubscript{2} MAPE @ 75\% & $<5\%$ (Prognosis) \\
 & VO\textsubscript{2} Stability & Low Variance \\
\midrule
\multirow{3}{*}{\textbf{Offline}} & \textbf{Hit Rate @ 20s} & $>\mathbf{90\%}$ (Precision) \\
 & Time Bias & $\approx 0$s \\
 & Bland-Altman LoA & Clinical limits \\
\bottomrule
\end{tabularx}
\end{table}

\section{Results}

\subsection{Offline Precision: Clinical Equivalence}
In the offline diagnostic setting, \modelname{} demonstrated high agreement with expert consensus. The cumulative hit-rate curve (Fig.~\ref{fig:offline_acc}) shows that 92\% of predictions fell within a $\pm$20s tolerance (2 bins) of the ground truth. Bland-Altman analysis revealed a mean bias of 0.8s, with limits of agreement narrower than reported inter-observer variability ($\pm$30s).

\subsection{Online Safety: The Zero-False-Alarm Standard}
For real-time monitoring, safety is paramount. Fig.~\ref{fig:online_safety} illustrates the distribution of trigger delays. Crucially, the "Early Trigger" region (negative delay) is virtually empty ($<$1.5\%), ensuring the model does not prematurely terminate tests. The mean trigger delay was 18s, which is physiologically acceptable given the persistence logic required to filter noise.

\subsection{Prognostic Value: Virtual Maximal Testing}
The \vopeak{} convergence plot (Fig.~\ref{fig:prognosis}) demonstrates the model's prognostic capability. The Mean Absolute Percentage Error (MAPE) drops below 5\% once 75\% of the test duration is completed. This suggests that for high-risk patients, a sub-maximal test (stopping at $\sim$75\% effort) combined with \modelname{} can reliably estimate functional capacity without inducing maximal cardiac stress.

\subsection{Ablation Study: The Role of Style Adaptation}
Removing the Style Encoder resulted in a significant increase in systemic bias (+14s on Center B), confirming that the input-driven adaptation effectively decouples device heterogeneity from physiological features.

\section{Discussion}

This study presents the first CPET analysis framework to explicitly decouple and optimize for the conflicting requirements of safety and precision. By leveraging meso-scale aggregation (10s bins) and Conditional Layer Normalization, \modelname{} overcomes the noise and heterogeneity inherent in multi-center respiratory data.

The physiological significance of the \textbf{Conformer backbone} is evident in its ability to handle heterophasic coupling; the attention mechanism naturally aligns the lagged ventilatory response with metabolic events. Furthermore, the \textbf{Dual-Head design} validates the feasibility of "Virtual Maximal Testing," potentially transforming CPET protocols for heart failure and perioperative populations.

\section{Conclusion}
\modelname{} establishes a new technical standard for automated CPET interpretation. It provides a clinically safe, diagnostically precise, and universally applicable tool that requires minimal calibration, paving the way for large-scale, standardized cardiopulmonary phenotyping.

%% ===== BIBLIOGRAPHY =====
\begin{thebibliography}{99}
\bibitem{guazzi2016} Guazzi M, et al. 2016 European Guidelines on cardiovascular disease prevention in clinical practice. \textit{Eur Heart J}. 2016;37:2315-2381.
\bibitem{wasserman2012} Wasserman K, et al. \textit{Principles of Exercise Testing and Interpretation}. 5th edn. Lippincott Williams \& Wilkins; 2012.
\bibitem{beaver1986} Beaver WL, et al. A new method for detecting anaerobic threshold by gas exchange. \textit{J Appl Physiol}. 1986;60:2020-2027.
\bibitem{gulati2020} Gulati A, et al. Conformer: Convolution-augmented Transformer for Speech Recognition. \textit{Interspeech}. 2020.
\end{thebibliography}

\newpage
\onecolumn
\appendix
\section{The Unified CPET Data Standard}
\label{app:schema}

The following tables define the \texttt{CPETx\_Standard} schema used to harmonize multi-center data. All downstream AI models consume data strictly adhering to this interface.

\subsection{Time-Series Schema (Breath-by-Breath / Continuous)}

% 此处使用了 longtable 环境,必须在 preamble 引入 \usepackage{longtable}
\begin{longtable}{l l l p{6cm}}
\caption{Time-Series Variable Definitions} \label{tab:ts_schema} \\
\toprule
\textbf{Standard Name} & \textbf{Unit} & \textbf{Type} & \textbf{Description} \\
\midrule
\endfirsthead
\multicolumn{4}{c}%
{\tablename\ \thetable\ -- \textit{Continued from previous page}} \\
\toprule
\textbf{Standard Name} & \textbf{Unit} & \textbf{Type} & \textbf{Description} \\
\midrule
\endhead
\hline \multicolumn{4}{r}{\textit{Continued on next page}} \\
\endfoot
\bottomrule
\endlastfoot

% === Temporal ===
\multicolumn{4}{l}{\textit{Temporal Dynamics}} \\
\texttt{Time} & mm:ss & string & Elapsed time from test start. \\
\texttt{Phase\_Time} & mm:ss & string & Time elapsed within current phase. \\
\texttt{Time\_Relative} & s & float & Relative time pointer (0.0 to T). \\

% === Gas Exchange ===
\multicolumn{4}{l}{\textit{Gas Exchange \& Metabolism}} \\
\texttt{VO2} & mL/min & float & Absolute Oxygen Consumption. \\
\texttt{VO2\_kg} & mL/kg/min & float & Relative Oxygen Consumption. \\
\texttt{VCO2} & mL/min & float & Carbon Dioxide Production. \\
\texttt{RER} & ratio & float & Respiratory Exchange Ratio ($VCO_2/VO_2$). \\
\texttt{METS} & MET & float & Metabolic Equivalents. \\

% === Ventilation ===
\multicolumn{4}{l}{\textit{Ventilation}} \\
\texttt{VE} & L/min & float & Minute Ventilation (BTPS). \\
\texttt{Bf} & 1/min & float & Breath Frequency. \\
\texttt{VT} & L & float & Tidal Volume. \\
\texttt{PetO2} & mmHg & float & End-tidal $PO_2$. \\
\texttt{PetCO2} & mmHg & float & End-tidal $PCO_2$. \\
\texttt{VE\_VO2} & ratio & float & Ventilatory Equivalent for $O_2$. \\
\texttt{VE\_VCO2} & ratio & float & Ventilatory Equivalent for $CO_2$. \\

% === Cardiovascular ===
\multicolumn{4}{l}{\textit{Cardiovascular \& ECG}} \\
\texttt{HR} & bpm & int & Heart Rate. \\
\texttt{VO2\_HR} & mL/beat & float & Oxygen Pulse. \\
\texttt{SpO2} & \% & float & Oxygen Saturation. \\
\texttt{BP\_Syst} & mmHg & int & Systolic Blood Pressure. \\
\texttt{BP\_Diast} & mmHg & int & Diastolic Blood Pressure. \\
\texttt{ST\_[Lead]} & mV & float & ST-segment deviation (Leads I-V6). \\

% === Load ===
\multicolumn{4}{l}{\textit{Ergometer \& Protocol}} \\
\texttt{Power\_Load} & W & float & External Work Rate. \\
\texttt{RPM} & r/min & int & Pedaling Cadence. \\
\texttt{Load\_Phase} & cat & int & 0:Rest, 1:Warmup, 2:Exercise, 3:Recovery. \\

\end{longtable}

\subsection{Summary Metrics Schema (Report Level)}
\begin{table}[h]
\centering
\caption{Key Prognostic Summary Metrics}
\label{tab:summary_schema}
\begin{tabularx}{\textwidth}{l l X}
\toprule
\textbf{Standard Name} & \textbf{Unit} & \textbf{Description} \\
\midrule
\multicolumn{3}{l}{\textit{Aerobic Capacity}} \\
\texttt{Peak\_VO2\_kg} & mL/kg/min & The highest 20-30s average $VO_2$ achieved. \\
\texttt{Peak\_VO2\_Predicted} & mL/min & Predicted value based on Wasserman/Hansen equations. \\
\texttt{Peak\_METS} & MET & Peak functional capacity. \\
\midrule
\multicolumn{3}{l}{\textit{Anaerobic Threshold (AT)}} \\
\texttt{VO2\_at\_AT} & mL/min & $VO_2$ at the moment of Anaerobic Threshold. \\
\texttt{HR\_at\_AT} & bpm & Heart Rate at AT. \\
\texttt{Time\_at\_AT} & mm:ss & Timestamp of the AT event. \\
\midrule
\multicolumn{3}{l}{\textit{Ventilatory Efficiency}} \\
\texttt{VE\_VCO2\_Slope} & ratio & Linear regression slope of $VE$ vs $VCO_2$. \\
\texttt{OUES} & - & Oxygen Uptake Efficiency Slope. \\
\texttt{Peak\_PetCO2} & mmHg & Maximum end-tidal $CO_2$ pressure. \\
\bottomrule
\end{tabularx}
\end{table}

\subsection{Metadata \& Calibration Schema}
\begin{table}[h]
\centering
\caption{Subject and Environmental Metadata}
\label{tab:meta_schema}
\begin{tabularx}{\textwidth}{l l X}
\toprule
\textbf{Category} & \textbf{Fields} & \textbf{Note} \\
\midrule
\textbf{Subject} & \texttt{Subject\_ID}, \texttt{Age}, \texttt{Gender}, \texttt{Height\_cm}, \texttt{Weight\_kg} & Essential for predicting normative values. \\
\midrule
\textbf{Exam} & \texttt{Exam\_ID}, \texttt{Date}, \texttt{Protocol\_Name}, \texttt{Ergometer\_Type} & Defines the physical context of the test. \\
\midrule
\textbf{Env} & \texttt{Pressure\_Barometric}, \texttt{Temp\_Ambient}, \texttt{RH\_Ambient} & Used for STPD/BTPS gas correction. \\
\bottomrule
\end{tabularx}
\end{table}

\end{document}
